
% CGC Lyman-alpha Section - THESIS

\section{Lyman-$\alpha$ Forest Constraints}
\label{sec:lyalpha_constraints}

The Lyman-$\alpha$ forest provides a crucial independent test of the CGC framework 
at high redshift ($2.2 < z < 4.2$). We present two complementary analyses:

\subsection{Analysis A: Primary Parameter Estimation}

Our primary parameter estimation uses Planck 2018 CMB temperature power spectrum, 
BOSS DR12 BAO measurements, redshift-space distortion growth data, and local H$_0$ 
measurements. This yields:
\begin{equation}
    \mu = 0.4113 \pm 0.0440 \quad (9.4\sigma \text{ detection})
\end{equation}

With these parameters, the predicted CGC enhancement at Lyman-$\alpha$ scales 
is approximately 136\% at $z=3$, which exceeds the DESI systematic 
uncertainty of $\sim$7.5\%.

\subsection{Analysis B: Joint Fit Including Lyman-$\alpha$}

When eBOSS Lyman-$\alpha$ flux power spectrum data is included in the likelihood, 
the CGC coupling is constrained to:
\begin{equation}
    \mu = 0.0449 \pm 0.0186 \quad (2.4\sigma \text{ detection})
\end{equation}

This represents a factor of 9.1$\times$ reduction in $\mu$, bringing 
the predicted Lyman-$\alpha$ enhancement to 6.5\%, which is within 
observational uncertainties.

\subsection{Physical Interpretation}

The significant shift in $\mu$ between analyses demonstrates that:

\begin{enumerate}
    \item Lyman-$\alpha$ data provides a strong constraint on CGC at high redshift
    \item The CGC framework predictions at Lyman-$\alpha$ scales are testable (falsifiable)
    \item When Lyman-$\alpha$ is included, CGC still provides a statistically 
          significant improvement over $\Lambda$CDM
\end{enumerate}

The reduction in H$_0$ tension resolution from 49\% to 5\% 
reflects the trade-off between fitting low-$z$ tension data and respecting 
high-$z$ Lyman-$\alpha$ constraints.

\begin{figure}[htbp]
    \centering
    \includegraphics[width=\textwidth]{plots/cgc_thesis_lyalpha_transparency.pdf}
    \caption{Comparison of CGC parameter constraints with and without Lyman-$\alpha$ 
    forest data. (a) Posterior distribution of the CGC coupling $\mu$. 
    (b) Predicted CGC enhancement at Lyman-$\alpha$ scales. 
    (c) Key parameter comparison.}
    \label{fig:lyalpha_comparison}
\end{figure}

